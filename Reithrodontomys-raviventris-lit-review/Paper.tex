%        File: Paper.tex
%     Created: Sat Mar 14 09:00 AM 2015 P
% Last Change: Sat Mar 14 09:00 AM 2015 P
%
\documentclass[a4paper]{article}
\author{Jeremy Wong}
\title{\textit{Reithrodontomys raviventris}\\ \textsc{A Literature Review}}

\begin{document}
\maketitle

The salt-marsh harvest mouse, also known as \textit{Reithrodontomys Raviventris}, is an endangered species endemic to the San Francisco Bay Area, California (Geissel \textit{et al}., 1988). They are able to survive in environments where water and vegetation contain a relatively high concentration of salt (Shellhammer, 1989). There are two subspecies of \textit{Reithrodontomys raviventris}: the northern \textit{Reithrodontomys raviventris halicoetes} and the southern \textit{Reithrodontomys raviventris raviventris} (Shellhammer, 1982). This paper will provide a cursory overview of the literature regarding the salt-marsh harvest mouse, including topics such as: physiology, habitat, ancestral lineage, reproduction and diet.

The salt marsh harvest mouse is small, only 2.75 to 3 inches lengthwise. Its fur is buff to brown, and may have a dark stripe running down its back (epa.gov, 2015). Despite its name, \textit{Reithrodontomys raviventris} often do not have any red coloration. They weigh 7.6 to 8.5 grams (Shellhammer, 1982). There are some differences between the physical characteristics of the two subspecies, \textit{R. r. raviventris} and \textit{R. r. halicoetes}. The \textit{R. r. raviventris} is of a darker hue, while the \textit{R. r. halicoetes} is lighter. The ears of both subspecies are dark, and some may have tufts of hairs behind the ears (Shellhammer, 1982).

The \textit{R. raviventris} is of a calm temperament, especially when compared to one of its relative species the \textit{R. megalotis} (western harvest mouse) (Shellhammer, 1982). Salt-marsh harvest mice are both solitary (Fisler, 1971) and nocturnal (Smith \textit{et al}., 2014), although they are sometimes active during the day. Neither subspecies are known to be burrowers, and they prefer to nest (Shellhammer 1982). They display considerable behavioural plasticity, and are able to adapt to somewhat novel conditions when they are presented. They have been found (via trapping) to be willing to traverse expanses of open ground with little cover. When found in diked marshes, \textit{R. raviventris} use grasses more often than when found in tidal marshes (Geissel \textit{et al}., 1988). In dike marsh environments, salt marsh harvest mice have been found to exhibit lower rates of diurnal movement than their tidal marsh counterparts. Smith \textit{et al}., suggest two possible explanations for this behaviour: (1) the lack of tidal influence (inundation) has increased predators' access to the marsh, thereby forcing mice into refuge during the day and (2) the lack of tidal influence has increased the mice's access to resources throughout all times of any 24-hour day, in effect, making nocturnal foraging and caching endeavors more fruitful (Smith \textit{et al}., 2014).

Unlike many other mammals, both subspecies of \textit{R. raviventris} employ an unusual reproductive strategy. Breeding seasons are significantly longer for male members than female members. Males are active between April and September. Female \textit{R. r. halicoetes} are active March to November, while their \textit{R. r. raviventris} counterparts are only active from May to November. In general, there also seem to be more male \textit{R. raviventris} than female \textit{raviventris}--about 60\% of each subspecies of the salt-marsh harvest mice are male. It is also likely that these two facts are related; with more males, the \textit{raviventris} are able to breed even when a female is fertile at an unfavorable time (as there will likely be one or more males also ready at the same time) (Fisler, 1971).

The \textit{raviventris}' persistence in salt marshes is indicative of its ability to survive on a diet high in salt, a quality considered unusual for a mammal. They are able to consume highly salinated water without any adverse effect on body weight or any occurences of diarrhea. In particular, the \textit{R. r. halicoetes} is able to consume sea water, whereas the \textit{R. r. raviventris} drinks water with salt content between fresh and sea water--though, they tend to prefer water with salinities \textit{closer} to sea than fresh water (Haines, 1964; Shellhammer, 1982; Padgett-Flohr and Isakson, 2003). The \textit{R. r. raviventris} cannot survive on water with higher salt concentrations for extended lengths of time. The \textit{R. r. halicoetes} prefers fresh water, but is able to survive for up to 13 months on sea water. The kidney of the \textit{raviventris} produces a concentrated urine, a requirement for survival on salt water. Salt marsh harvest mice are able to process both with equal efficiency, and thus, the salinity of its water source does not appear to negatively affect lifespan (Haines, 1964).

This ability to consume highly salinated water gives them a competitive edge over other species found in salt-marshes, particularly the \textit{Microtus californicus} and \textit{R. megalotis}. This advantage manifests in environments where salinated water is the only water available, or where there are alternating cycles of fresh and salinated water (Geissel \textit{et al}., 1988). In the latter case, population of \textit{R. raviventris} would like fluctuate--rising as water salinity rises, and falling as water salinity falls.

The \textit{R. r. raviventris} can be found in salt marsh habitats, whereas the \textit{R. r. halicoetes} can be found in brackish marshes. It is worth noting that the former subspecies is afforded an advantage over their relatives by this difference--salt marshes offer better coverage, and thus, better protection from predation and other environmental risks (Fisler, 1971). Specifically, the greatest non-predatorial, environmental risk for the \textit{R. raviventris} is flooding. 99\% of the time, harvest mice will use vegatation to escape (vertically) from floods. Only in the remaining 1\% of cases do harvest mice travel upland to escape flooding (Smith \textit{et al}., 2014).

Predatorial threats include the red fox (\textit{Vulpes vulpes}), grey fox (\textit{Urocyon cinereoargenteus}), domestic cat (\textit{Felix domestica}), skunk (\textit{Mephitis mephitis}) and racoon (\textit{Procyon lotor}) (Sacremento Fish and Wildlife Office staff, 2010).

The salt marsh harvest mouse subsists on seeds, flowers, cactus fruit, green sprouts and various invertebrates. It is very likely that the diets of both subspecies of \textit{R. raviventris} consist of mostly vegetation (Shellhamer 1982), which may very well be why they have longer intestines than their (previously alleged) forebears, the \textit{R. megalotis}. Both subspecies appear to thrive in presence of halophytic vegetation, specifically, a mixture that is mostly of pickleweed and some alkalai heath, as well as a smaller percentage of other marsh vegetation (Geissel \textit{et al}., 1988; Bias \textit{et al}., 2006). In addition, their fur does not collect water as quickly as that of the \textit{megalotis}, which make them better swimmers (Shellhammer, 1982).

The salt marsh harvest mouse's status as a critically endangered species is an aggregate of many factors, all of which are the direct or indirect consequences of environmental change. Most of these factors work in concert against the population size of the \textit{R. raviventris}. Where a single one of these factors alone would prove to be easily managable by the \textit{raviventris}' reproductive strategy, a set of them can exert a level pressure far greater than the sum of its parts. The root cause of the salt-marsh harvest mouse's critical endangerment is change to its environment. In particular, fragmentation of its salt marsh habitat into smaller isolated marshes. This has the effect of limiting the space, and thus, the availability of all resources available to the salt-marsh harvest mouse, including: total cover from predation, total vertical escape paths from flooding (which also increases the chance that horizontal escape--upland travel--must be used), limited food, possibly limited nesting area, and lower chances of coming into contact with other members of the population for breeding. It's worth clarifying further how fragmentation affects the salt-marsh  harvest mouse. During floods, the \textit{raviventris} prefers to escape by simply climbing up any tall vegetation available. In the event that none is found, escaping upland to higher ground is the only remaining option. This usually means travel over vast expanses of open ground, with no cover from predation. In the event that they must travel over nearby roads or highways, they are also at risk of predation by automobile.

Continued existence of the salt marsh harvest mouse depends upon proper maintenance and management of its habitat. A large proportion of tidal marshes have been destroyed, and the remaining tidal marshes do not properly support populations of \textit{raviventris} due to: fragmentation (presumably due to diking), backfilling, subsidence, changes in vegetation, or habitat management for a different species \textit{entirely}.(Geissel \textit{et al}., 1988)

Most literature report that the \textit{R. raviventris}' reproductive strategy is not a limiting factor to its population size when compared to other mortality factors (Sacremento Fish and Wildlife Office staff, 2010). This implies that their solitary nature also is not a limiting factor to population size when compared to other mortality factors.

For a sense of completion, it is worth discussing the evolutionary lineage of the \textit{R. raviventris}. A large body of older literature treats the \textit{R. megalotis} as the parent species of the \textit{R. raviventris}, primarily using morphological, physiological and behavioural comparisons (Shellhammer, 1967; Hood 1984). More contemporary results suggest that, on a genetic level, the \textit{R. raviventris} is more closely related to another species, \textit{R. montanus}, yet at the same time, also distinct from it. (Nelson K \textit{et al}., 1984; Bell \textit{et al}., 2001). Bell's cytochrome-\textit{b} sequence datum show a 13.5\% genetic distance between the \textit{raviventris} and \textit{montanus}; and a 14.75\% genetic distance between the \textit{raviventris} and \textit{megalotis}. While the differences between the two distances is small, it is certainly large enough to make a reasonable claim that the \textit{raviventris} is more closely related to the \textit{montanus} than the \textit{megalotis}. This is certainly a better basis for deciding evolutionary lineage, since many species of \textit{Reithrodontomys} have a significant number of morphological similarities (Hood, 1984).

\section*{Works Cited}

Bias MA, Morrison ML. Habitat Selection of the Salt Marsh Harvest Mouse and Sympatric Rodent Species. Journal of Wildlife Management. 2006; 70(3): 732-742\\
Environmental Protection Agency. Endangered Species Facts; Salt Marsh Harvest Mouse. http://www.epa.gov/espp [accessed 2015 March 20]\\
Fisler G. Age Structure and Sex Ratio in Populations of Reithrodontomys. American Society of Mammalogists. 1971; 52(4): 653-662\\
Geissel W, Shellhammer H, Harvey T. The Ecology of the Salt-Marsh Harvest Mouse (\textit{Reithrodontomys raviventris}) in a Diked Salt Marsh. Journal of Mammalogy. 1988; 69(4): 696-703\\
Haines H. Salt Tolerance and Water Requirements in the Salt-Marsh Harvest Mouse. Physiological Zoology. 1964; 37(3): 266-272\\
Hood C, Robbins L, Baker R, Shellhammer H. Chromosomal Studies and Evolutionary Relationships of an Endangered Species, \textit{Reithrodontomys raviventris}. American Society of Mammalogists. 1984; 65(4): 655-667\\
Nelson K, Baker R, Shellhammer H, Chesser R. Test of Alternative Hypotheses Concerning the Origin of \textit{Reithrodontomys Raviventris}: Genetic Analysis. Journal of Mammalogy. 1984; 65(4): 668-673\\
Padgett-Flohr G, Isakson L. A Random Sampling of Salt Marsh Harvest Mice in a Muted Tidal Marsh. Journal of Wildlife Management. 2003; 67(3): 646-653\\
Sacremento Fish and Wildlife Office staff. Salt marsh harvest mouse (\textit{Reithrodontomys raviventris}). U.S. Fish and Wildlife Service 5-Year Review. 2010\\
Shellhammer H. Cytotaxonomic Studies of the Harvest Mice of the San Francisco Bay Region. Journal of Mammalogy. 1967; 48(4): 549-556\\
Shellhammer H. \textit{Reithrodontomys raviventris}. American Society of Mammalogists. 1982; 169: 1-3\\
Shellhammer H. Salt Marsh Harvest Mic, Urban Development, and Rising Sea Levels. Society for Conservation Biology. 1989; 3(1): 59-65\\
Smith K, Barthman-Thompson L, Gould W, Mabry K. Effects of Natural and Anthropogenic Change on Habitat Use and Movement of Endandgered Salt Marsh Harvest Mice. PLoS ONE. 2014; 9(10)\\
\end{document}
